\section{Psalm 46 Comments}

\subsection{Numeric Nuggets}
\textbf{13:} Verses 6, 7, 8, and 11 each contain 13 unique words.\\
\noindent \textbf{7:} The word ``God'' is used 7 times.\\
\noindent \textbf{5:} The word ``earth'' is used 5 times.



\subsection{Psalm 46:1}
One author notes that verse 1 is inscribed on the tombstone of Martin Luther.\cite{Ruckman1992PsalmsV1} What matters to a person in trouble is not past help, and not future help, but present help.
	
\subsection{Psalm 46:2}
The verse describes things that will happen at the Second Advent. All the mountains in the world will be flattened (the eventual Flat Earth that some believe in now) except for the mountain where Jerusalem sites. This place will be the highest point on Earth during the Millennium. 
	
\subsection{Psalm 46:3}
See verses like Joel 3:16 and Haggai 2:6-7 speaking of the mountains shaking.\footnote{\textbf{Joel 3:16-17} - The LORD also shall roar out of Zion, and utter his voice from Jerusalem; and the heavens and the earth shall shake: but the LORD will be the hope of his people, and the strength of the children of Israel. [17] So shall ye know that I am the LORD your God dwelling in Zion, my holy mountain: then shall Jerusalem be holy, and there shall no strangers pass through her any more.}\footnote{\textbf{Haggai 2:6-7} - For thus saith the LORD of hosts; Yet once, it is a little while, and I will shake the heavens, and the earth, and the sea, and the dry land; [7] And I will shake all nations, and the desire of all nations shall come: and I will fill this house with glory, saith the LORD of hosts.}
	
\subsection{Psalm 46:4}
This river is spoken of in Ezekiel 47. It will be in the midst of Jerusalem in the Millennium and in New Jerusalam in eternity, described in Revelation 22:1-2.\footnote{\textbf{Revelation 22:1-2} - And he shewed me a pure river of water of life, clear as crystal, proceeding out of the throne of God and of the Lamb. [2] In the midst of the street of it, and on either side of the river, was there the tree of life, which bare twelve manner of fruits, and yielded her fruit every month: and the leaves of the tree were for the healing of the nations.}

\subsection{Psalm 46:7}
Note the word ``Selah,'' which always indicates the tribulation context.  See the examples when the God of Jacob  protected Jacob:
\begin{compactenum}
	\item Protected from being killed by his brother in Genesis 27:41-46.
	\item Protected from being ignored during the blessing in Genesis 25:33-34.
	\item Protected from being cheated by Laban in Genesis 31:7.
	\item Protected from getting involved in a war over Dinah in Genesis 34:30.
	\item Protected from an early death -- lived 130 years (Genesis 47:9)
	\item Protected during the Great Tribulation, the context of Psalm 46.\cite{Ruckman1992PsalmsV1}
\end{compactenum}

\subsection{Psalm 46:8}
Compare this verse with Psalm 145:9.\footnote{\textbf{Psalm 145:9} - The LORD is good to all: and his tender mercies are over all his works.} Is this a contradiction? These works, here, involve making the Earth desolate. But they are merciful. Verse  9 explains that the Lord's ``war to end all wars'' will put an end to mankind's wars.

\subsection{Psalm 46:10}
In this age, the Lord is exalted locally, in churches where the Gospel is preached, where believers live lives to glorify him, and anyway God's truth is followed. Universally, that is worldwide, this will be true in the Millennium.
