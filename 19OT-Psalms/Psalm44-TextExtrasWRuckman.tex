\chapter{Psalm 44}
\footnote{\textcolor[rgb]{0.00,0.25,0.00}{\hyperlink{TOC}{Return to end of Table of Contents.}}}\textcolor[rgb]{0.00,0.00,1.00}{To the chief Musician for the sons of Korah, Maschil.}\\
\\
\textcolor[rgb]{0.00,0.00,1.00}{We have heard with our ears, O God, our fathers have told us, \emph{what} work thou didst in their days, in the times of old.}\index[AWIP]{We!Psalms!Psa 044:001}\index[AWIP]{have!Psalms!Psa 044:001}\index[AWIP]{heard!Psalms!Psa 044:001}\index[AWIP]{with!Psalms!Psa 044:001}\index[AWIP]{our!Psalms!Psa 044:001}\index[AWIP]{ears!Psalms!Psa 044:001}\index[AWIP]{O!Psalms!Psa 044:001}\index[AWIP]{God!Psalms!Psa 044:001}\index[AWIP]{our!Psalms!Psa 044:001 (2)}\index[AWIP]{fathers!Psalms!Psa 044:001}\index[AWIP]{have!Psalms!Psa 044:001 (2)}\index[AWIP]{told!Psalms!Psa 044:001}\index[AWIP]{us!Psalms!Psa 044:001}\index[AWIP]{\emph{what}!Psalms!Psa 044:001}\index[AWIP]{work!Psalms!Psa 044:001}\index[AWIP]{thou!Psalms!Psa 044:001}\index[AWIP]{didst!Psalms!Psa 044:001}\index[AWIP]{in!Psalms!Psa 044:001}\index[AWIP]{their!Psalms!Psa 044:001}\index[AWIP]{days!Psalms!Psa 044:001}\index[AWIP]{in!Psalms!Psa 044:001 (2)}\index[AWIP]{the!Psalms!Psa 044:001}\index[AWIP]{times!Psalms!Psa 044:001}\index[AWIP]{of!Psalms!Psa 044:001}\index[AWIP]{old!Psalms!Psa 044:001}\index[NWIV]{25!Psalms!Psa 044:001}\index[PNIP]{God!Psalms!Psa 044:001}\footnote{[RUCKMAN] ``What work thou didst in their days'' (vs. 1) is explained in verses 2 and 3. It is the events recorded in Exodus and Joshua. This is called a ``parable'' in Psalm 78:2. It will take place again. The writer of Hebrews is so sure that it will take place again that he likens the Tribulation generation to the generation that came up under Joshua and Caleb (Hebrews 3–-4). Verses 2 and 3 are clear to anyone who reads Deuteronomy 1-–10; Joshua 1-–10; and Judges 1–-4. Devotionally, one may ask ``What work is God doing these days?'' Further, we see that secondhand knowledge is better than no knowledge, but firsthand knowledge is still the best (see Joshua 2:9 and Judges 7:9–-11). Spiritually, we may observe that victory over the flesh -— typified by Israel’s victories over Amalek, the Moabites, Ammonites, and Amorites -— is through Jesus Christ by grace, not self-effort: ``neither did their own arm save them''(vs. 3). \cite{Ruckman1992Psalms} }


[8] \footnote{[RUCKMAN] Note that the nation is involved in the individual petition. God may be the King of the writer (“my King, O God,” vs. 4), but what follows is ``deliverances for Jacob'': ``we...we...us...us...us...we...'' (vss. 5, 7, and 8). At the Advent Israel does triumph over her enemies; she does tread them down (vs. 5); she is saved (vs. 7), and her enemies are put to shame; then Israel will ``boast all the day long'' about God (vs. 8). And right here, at this point (vs. 8), the Lord suddenly divides the word of truth (exactly as He does it again in Psalm 89:37–-38), completely confounding every “scientific exegete,” “qualified authority,” and “godly, militant Fundamentalist” who ever messed with the Book. \cite{Ruckman1992Psalms}  }

[16] \footnote{[RUCKMAN] Mighty quick “reversal,” wouldn’t you say? When did this happen? First or 2 Samuel, 1 or 2 Kings, 1 or 2 Chronicles? It didn’t. Kroll (as Afman, Martin, Price, Horton, Hobbs, Kutilek, Dell, Sherman, Henderson, and Sumner) simply doesn’t know what he is talking about. Dummelow doesn’t even try to wet his lips; he has dried up completely. Jamieson, Fausset, and Brown don’t know what to do with the mess, for all indications are that verses 9–13 apply to the Babylonian Captivity. Some “Edomite” has run into Jamieson’s remarks at verse 15, and he has the whole scene up in the time of Jeremiah. How this matches verse 8 is impossible to conjecture unless verses 1–8 were written by David in 1000 B.C. with someone adding verses 9–-11 after Zedekiah or Jeconiah (600 B.C.) Verses 9 through 16 describe Israel in the Church Age and at the end of the Tribulation. Consider: (1) The Jews have no armies from A.D. 70 to 1945 (vs. 9); (2) In the Tribulation they will lose the battles in Palestine to the Antichrist (vs. 10); and (3) They were scattered in the Church Age and they will be sold for slaves in the Tribulation (vs. 12). Verse 13 applies from A.D. 70 to 1990 plus, as does verse 14. The Psalmist speaks for his nation in the first person in verse 15. Notice this eyewitness of the historical event which foreshadowed all this, not only spoke in the first person for the nation, but for the city of Jerusalem (see Lamentations 1:15, 19). ``The enemy and avenger'' (vs. 16) is identified so clearly in Revelation 12:9 and 13:6 that only ``the original autographs'' and the ``Dead Sea Scroll'' could keep you from finding the truth. The blasphemer of Revelation 13:6 is in Jerusalem, which now becomes ``Sodom and Egypt'' (Rev. 11:8) and is literally blaspheming, with the ``names of blasphemy'' (Rev. 17:3) on his forehead. \cite{Ruckman1992Psalms} }

[19] \footnote{[RUCKMAN] The reference to ``\textbf{shadow of death}'' puts Psalm 44 in a tribulation context. See Job 3:5, Job 10:21, Job 10:22, Job 12:22, Job 16:16, Job 24:27, Job 28:3, Job 34:22, Job 38:17, Psalm 23:4, here, Psalm 107:10, Psalm 107:14, Isaiah 9:2, Jeremiah 2:6, Jeremiah 13:16, Amos 5:8, Matthew 4:16, and Luke 1:79. \cite{Ruckman1992Psalms}  }

[22] \footnote{[RUCKMAN] Panic. Disaster. Into the dumpster go three thousand scholars with fifteen hundred years of education in back of them along with all of their Coptic, Syriac, Greek, and Hebrew texts. The causes for the first captivity (Solomon to Jeremiah) are not even remotely suggested in the passage. The Jews here who are being sold into slavery, tormented, ridiculed, blasphemed, and eaten (vs. 22)  \cite{Ruckman1992Psalms}   :\begin{compactenum}
\item Did not forget God (vs. 17): Jeremiah’s generation did. 
\item Did not deal falsely in the covenant (vs. 17): Jeremiah’s generation did.
\item Did not turn back in their hearts (vs. 18): Jeremiah’s generation did.
\item Did not turn their steps aside (vs. 18): Jeremiah’s generation did.
\item Did not forget the name of God (vs. 20): Jeremiah’s generation did.
\item Did not stretch out their hands to a strange god (vs. 20): Jeremiah’s generation did.
\end{compactenum} 
They did ALL those things and more, too (Jer. 1–40).  The Jews who were killed in the Book of Judges were not killed “for thy sake” (vs. 22), nor were the Jews in Sennacherib’s invasion, nor Nebuchadnezzar’s invasion, nor when Titus destroyed Jerusalem, nor any time since, unless it was a Christ-rejecting Jew killed by a Catholic, or a Communist, or a Moslem. The “sheep for the slaughter” are sacrificial sheep slaughtered at an altar in a temple (Rev. 6:9). They are decapitated (Rev. 20:4) and eaten (Isa. 6:13).}

[26] \footnote{[RUCKMAN] Compare with \textbf{Lamentations 5:20--22} -- Wherefore dost thou forget us for ever, and forsake us so long time? [21] Turn thou us unto thee, O LORD, and we shall be turned; renew our days as of old. [22] But thou hast utterly rejected us; thou art very wroth against us. \cite{Ruckman1992Psalms}  }



